For most of early Greek history, there was no context of ``Greece" as a political or cultural entity. Therefore, it is useful to keep the wider context of the Mediterranean in mind while examining the development of Greek civilization.

Ancient Greek history is typically divided into five periods:
\begin{enumerate}
\item The Bronze Age, lasting from roughly 3000 to 1200 BCE, in which the Aegean was dominated by Minoan civilization.

\item The Iron Age, lasting from 1200 to c. 800 BCE, in which the Greeks slowly recovered from the power vacuum left by the fall of the Minoans.

\item The Archaic Period, lasting from c. 800 to 479 BCE, in which the Greeks began to formulate a sense of ``Greekness", culminating with the emergence of Greece as a more-or-less unified power after defeating a Persian invasion in the Greco-Persian War.

\item The Classical Period, lasting from 479 to 323 BCE, in which the Greeks entered a golden age, ending in the rise and fall of Alexander the Great.

\item The Hellenistic Period, lasting from 323 to 31 BCE, in which Greece slowly declined in inflence until being subsumed into the Roman Empire.
\end{enumerate}
